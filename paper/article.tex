%% LyX 2.0.6 created this file.  For more info, see http://www.lyx.org/.
%% Do not edit unless you really know what you are doing.
\documentclass[english]{article}
\usepackage[T1]{fontenc}
\usepackage[latin9]{inputenc}
\usepackage[letterpaper]{geometry}
\geometry{verbose,tmargin=2.65cm,bmargin=2.65cm,lmargin=2.65cm,rmargin=2.65cm}
\setlength{\parindent}{0bp}
\usepackage{float}
\usepackage{amsthm}
\usepackage{amsmath}

\makeatletter
%%%%%%%%%%%%%%%%%%%%%%%%%%%%%% Textclass specific LaTeX commands.
\numberwithin{equation}{section}
\numberwithin{figure}{section}

\makeatother

\usepackage{babel}
\begin{document}

\title{This is Article}


\author{Christian Poveda Ruiz}

\maketitle

\section{Introduction}

The Navarro-Frenk-White mass profile can be normalized respect to
the virial mass and written in function of the normalized radius respect
to the virial radius $r_{norm}$ and the concentration $c$ 

\[
m_{nfw}\left(r_{norm},c\right)=A\left[\log\left(1+r_{norm}c\right)-\left(\frac{1}{r_{norm}c}+1\right)^{-1}\right]
\]
Where $A$ is a normalization constant such that $m_{nfw}\left(1,c\right)=1$


\section{State of Art}


\section{What did I do?}

In order to obtain a mass profile for each halo in funcion of the
radius, the center of each halo must be calculated. However the center
of mass will not give an accurate position for the center because
some haloes are highly irregular, is more convenient to take the gravitational
potential for each particle

\[
\phi\left(\boldsymbol{R}_{n}\right)=-\sum_{j\neq n}\frac{1}{\left\Vert \boldsymbol{R}_{n}-\boldsymbol{R}_{j}\right\Vert }
\]
Take $\boldsymbol{R}_{0}=\min\left(\phi\left(\boldsymbol{R}_{n}\right)\right)$
as the center then define new coordinates $\boldsymbol{R}_{n}^{\prime}=\boldsymbol{R}_{0}-\boldsymbol{R}_{n}$,
this $\boldsymbol{R}_{0}$ will be in the most relevant region for
the dynamics of each halo. Organizing the $\left\{ \boldsymbol{R}_{n}^{\prime}\right\} $
in crescent order, the accumulated mass for each $\boldsymbol{R}_{n}^{\prime}$
will be $M_{n}=nM$.

\begin{figure}[H]
\caption{FIXME: Mass profile of a halo}
\end{figure}


Then the virial radius is obtained and the particles that are beyond
the virial radius are removed. After that both the mass and the radius
get normalized respect to the mass and virial radius, we will call
these new variables $m_{n}$ and $r_{n}$. The last part of the process
consists in fitting the normalized NFW mass profile to the $m_{n}$
data using the Metropolis-Hastings algorithm to sample the Likelihood
${\cal L}\left(c\right)=\exp\left(-\chi^{2}\left(c\right)\right)$
where 
\[
\chi^{2}\left(c\right)=\sum\limits _{n}\left|m_{n}-m_{nfw}\left(r_{n},c\right)\right|^{2}
\]
And taking the values of $c$ where the Likelihood is maximum. Finally
we compare this results with the values obtaned by BDM, since the
number of halos obtained by FOF and BDM is different, each BDM halo
gets related with the nearest halo in FOF. We compute two concentrations,
one where the overdensity limit is $360\rho_{back}$ (like in BDMV)
an other where the overdensity limit is $740\rho_{back}$ (like in
BDMW).


\section{Results}


\subsection{Trial Problem with several Generated Halos}

To be sure that the results are reasonable, several tests were performed
generating halos from the Navarro-Frenk-White profile with known values
for $\rho_{0}$ and $r_{s}.$ The center of the generated halos is
$\boldsymbol{0}=\left(0,0,0\right)$ so we can determine how good
estimate is the minimum of the potential at the center of each halo.
Two types of tests were run: The first omitting the calculation of
the center of each halo and assuming that is $\boldsymbol{0}$ and
second by calculating the center as the minimum of the potential.

\begin{figure}[H]
\caption{FIXME: Test results}
\end{figure}



\subsection{Results from the simulations}


\subsection{What kind of implications (Low resolution simulations, other fitting
algorithms)}


\section{Conclusions}
\end{document}
