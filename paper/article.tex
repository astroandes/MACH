\documentclass[useAMS,usenatbib]{mn2e}
\usepackage{amsmath}
\usepackage{amssymb}
\usepackage{graphics}
\usepackage{graphicx}
\usepackage{epsfig} 
\usepackage{hyperref}
\def\be{\begin{equation}}
\def\ee{\end{equation}}
\def\ba{\begin{eqnarray}}
\def\ea{\end{eqnarray}}

% To highlight comments
\usepackage{color}
\definecolor{red}{rgb}{1,0.0,0.0}
\newcommand{\red}{\color{red}}
\definecolor{blue}{rgb}{0.1,0.3,0.9}
\newcommand{\blue}{\color{blue}}

\usepackage[normalem]{ulem}
\definecolor{darkgreen}{rgb}{0.0,0.5,0.0}

\newcommand{\documentname}{paper~}
\newcommand{\LCDM}{$\Lambda$CDM~}
\newcommand{\beq}{\begin{eqnarray}} 
\newcommand{\eeq}{\end{eqnarray}} 
\newcommand{\zz}{$z\sim 3$}

\newcommand{\apj}{ApJ} 
\newcommand{\apjs}{ApJS} 
\newcommand{\apjl}{ApJL} 
\newcommand{\aj}{AJ} 
\newcommand{\mnras}{MNRAS} 
\newcommand{\mnrassub}{MNRAS accepted} 
\newcommand{\aap}{A\&A} 
\newcommand{\aaps}{A\&AS} 
\newcommand{\araa}{ARA\&A} 
\newcommand{\nat}{Nature} 
\newcommand{\physrep}{PhR}
\newcommand{\pasp}{PASP}   
\newcommand{\pasj}{PASJ}   
\newcommand{\avg}[1]{\langle{#1}\rangle} 
\newcommand{\ly}{{\ifmmode{{\rm Ly}\alpha}\else{Ly$\alpha$}\fi}}
\newcommand{\hMpc}{{\ifmmode{h^{-1}{\rm Mpc}}\else{$h^{-1}$Mpc }\fi}} 
\newcommand{\hGpc}{{\ifmmode{h^{-1}{\rm Gpc}}\else{$h^{-1}$Gpc }\fi}} 
\newcommand{\hmpc}{{\ifmmode{h^{-1}{\rm Mpc}}\else{$h^{-1}$Mpc }\fi}} 
\newcommand{\hkpc}{{\ifmmode{h^{-1}{\rm kpc}}\else{$h^{-1}$kpc }\fi}}
\newcommand{\hMsun}{{\ifmmode{h^{-1}{\rm
        {M_{\odot}}}}\else{$h^{-1}{\rm{M_{\odot}}}$~}\fi}}  
\newcommand{\hmsun}{{\ifmmode{h^{-1}{\rm
        {M_{\odot}}}}\else{$h^{-1}{\rm{M_{\odot}}}$}\fi}}  
\newcommand{\Msun}{{\ifmmode{{\rm {M_{\odot}}}}\else{${\rm{M_{\odot}}}$}\fi}} 
\newcommand{\msun}{{\ifmmode{{\rm {M_{\odot}}}}\else{${\rm{M_{\odot}}}$}\fi}} 
\newcommand{\lya}{{Lyman$\alpha$~}}
\newcommand{\clara}{{\texttt{CLARA}}~}
\newcommand{\rand}{{\ifmmode{{\mathcal{R}}}\else{${\mathcal{R}}$ }\fi}} 
\newcommand{\hs}{{\hspace{1mm}}}
\newcommand{\muavg}{\vert\langle\cos\theta\rangle\vert}
% definition to produce a "less than or similar to" symbol
\def\lsim{~\rlap{$<$}{\lower 1.0ex\hbox{$\sim$}}}

% definition to produce a "greater than or similar to" symbol

% definition to produce a "greater than or similar to" symbol
\def\gsim{~\rlap{$>$}{\lower 1.0ex\hbox{$\sim$}}}

\begin{document}

\title[Bayesian halo concentration]{Measurement of dark matter halo concentrations using a bayesian approach}
\author[Poveda \& Forero-Romero]{
\parbox[t]{\textwidth}{\raggedright
  Christian Poveda$^{1}$ \&
  Jaime E. Forero-Romero$^{1}$
}
\vspace*{6pt}\\
$^{1}$Departamento de F\'{i}sica, Universidad de los Andes, Cra. 1
No. 18A-10, Edificio Ip, Bogot\'a, Colombia\\
}
\maketitle

\begin{abstract}
asd
\end{abstract}
\begin{keywords}
methods: numerical, galaxies: haloes, cosmology: theory, dark
matter
\end{keywords}


\section{Introduction}
\label{sec:introduction}


\citep{NFW}


\section{Basic properties of the NFW density profile}
\label{sec:basics}

The Navarro-Frenk-White density profile can be written as

\begin{equation}
\rho(r) = \frac{\rho_c\delta_c}{r/r_s(1+r/r_s)^2}, 
\end{equation}
%
where $\rho_c\equiv 3H^2/8\pi G$ is the Universe critical density,
$\delta_c$ is the halo dimensionless characteristic density and $r_s$
is known as the scale radius, the radius that marks the transition
between the two power law behaviour in the $\rho\propto r^{-1}$ for
$r<r_s$ and $\rho\propto r^{-2}$ for  $r>r_s$.  

We define the virial radius of a halo, $r_v$, as the boundary of the
spherical volume that encloses an average density of $\Delta_v$ times
the critical density of the Universe. The corresponding mass $M_{v}$
can be thus expressed as $M_{v} = \frac{4\pi}{3}\rho_c\Delta_v r_v^3$. 


The total mass enclosed within a radius $r$ can be computed to be:
\begin{equation}
M(<r) = 4\pi\rho_c\delta_c  r_s^3\left[\ln \left
  (\frac{r_s+r}{r}\right) - \frac{r}{r_s+r}\right].
\end{equation}
 
We can now define the concentration of the halo as $c=r_v/r_s$, the
dimensioless variable $x\equiv r/r_v$ and $m\equiv M(<r)/M_v$, which
allows us to express the total enclosed mass within a dimensionless
radius $x$ as:


\begin{equation}
m(<x) =
\frac{1}{A}\left[\ln\left(1+xc\right)-\left(\frac{xc}{xc+1}\right)\right],
\label{eq:profile}
\end{equation}
%
where 
%
\begin{equation}
A=\left[\ln\left(1+c\right)-\left(\frac{c}{c+1}\right)\right],
\end{equation}

meaning that the concentration is the only free parameter to
determine the density profile of the halo.


\begin{figure}
\caption{FIXME: Mass profile of a halo}
\end{figure}


\section{A Bayesian Approach to Halo Fitting}
\label{sec:method}


We proceed to find the value of the concentration parameter that
best describes the simulation data, following the model in
Eq. (\ref{eq:profile}). We use the integrated mass profile because it
allows us to use the data directly with the simulation without binning
the particle positions and estimating a density.

We construct the integrated mass profile by ranking the particles by
their increasing distance to the center of the halo. Once they are ranked,
the total mass at a radius $r_i$, increases by $m_p$, where $r_i$ is
the position of the $i$-th particle and $m_p$ is the mass of the
computational particle.  We define the center of the halo to be  at
the position of the particle with the lowest gravitational
potential. In the process of building the mass profile we discard the
particle at the center.

We stop the construction of the integrated mass profile once we arrive
at an average density of $\Delta_h\rho_c$. This radius marks the
virial radius and the virial mass. We divide the total mass enclosed
mass $M_i$ and the radii $r_i$ by these values to obtaine the
dimensionless variables $x_i$ and $m_i$. 

Using these positions and masses we define the following $\chi^2$ function

\begin{equation}
\chi^2(c) = \sum_{i}[m_i - m(< x_i;c)]^2, 
\end{equation}
%
where $m(<x_i;c)$ corresponds to the values in Eq.(\ref{eq:profile}) at
$x=x_i$ and a given value of the concentration parameter $c$.

Finally we use a Metropolis-Hastings algorithm to sample the likelihood
function distribution defined by ${\cal L}(c)=\exp(-\chi^2(c)/2)$ to
find the optimum value of $c$ and its associated uncertainty
$\sigma_c$. 

\section{Numerical Experiments}
Finally we compare this results with the values obtaned by BDM, since
the number of halos obtained by FOF and BDM is different, each BDM
halo gets related with the nearest halo in FOF. We compute two
concentrations, one where the overdensity limit is $360\rho_{back}$
(like in BDMV) an other where the overdensity limit is
$740\rho_{back}$ (like in BDMW). 

\subsection{Mock Data and Simulations}
\label{sec:data}
\section{Results}
\label{sec:results}

\subsection{Trial Problem with several Generated Halos}


Several halos with known values of $c$ were generated. These halos
were then processed to determine the acurracy of this method. The
number of particles and value of $c$ for each halo were generated
randomly.  
\begin{figure}[H]
\caption{FIXME: Test results}
\end{figure}



\subsection{Results from the simulations}


\subsection{What kind of implications (Low resolution simulations, other fitting algorithms)} 


\section{Discussion}
\label{sec:discussion}


\section{Conclusions}
\label{sec:conclusions}


\bibliographystyle{mn2e}
\bibliography{references}

\end{document}
