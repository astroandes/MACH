\documentclass[useAMS,usenatbib]{mn2e}
\usepackage{amsmath}
\usepackage{amssymb}
\usepackage{graphics}
\usepackage{graphicx}
\usepackage{epsfig} 
\usepackage{hyperref}
\def\be{\begin{equation}}
\def\ee{\end{equation}}
\def\ba{\begin{eqnarray}}
\def\ea{\end{eqnarray}}

% To highlight comments
\usepackage{color}
\definecolor{red}{rgb}{1,0.0,0.0}
\newcommand{\red}{\color{red}}
\definecolor{blue}{rgb}{0.1,0.3,0.9}
\newcommand{\blue}{\color{blue}}

\usepackage[normalem]{ulem}
\definecolor{darkgreen}{rgb}{0.0,0.5,0.0}

\newcommand{\documentname}{paper~}
\newcommand{\LCDM}{$\Lambda$CDM~}
\newcommand{\beq}{\begin{eqnarray}} 
\newcommand{\eeq}{\end{eqnarray}} 
\newcommand{\zz}{$z\sim 3$}

\newcommand{\apj}{ApJ} 
\newcommand{\apjs}{ApJS} 
\newcommand{\apjl}{ApJL} 
\newcommand{\aj}{AJ} 
\newcommand{\mnras}{MNRAS} 
\newcommand{\mnrassub}{MNRAS accepted} 
\newcommand{\aap}{A\&A} 
\newcommand{\aaps}{A\&AS} 
\newcommand{\araa}{ARA\&A} 
\newcommand{\nat}{Nature} 
\newcommand{\physrep}{PhR}
\newcommand{\pasp}{PASP}   
\newcommand{\pasj}{PASJ}   
\newcommand{\avg}[1]{\langle{#1}\rangle} 
\newcommand{\ly}{{\ifmmode{{\rm Ly}\alpha}\else{Ly$\alpha$}\fi}}
\newcommand{\hMpc}{{\ifmmode{h^{-1}{\rm Mpc}}\else{$h^{-1}$Mpc }\fi}} 
\newcommand{\hGpc}{{\ifmmode{h^{-1}{\rm Gpc}}\else{$h^{-1}$Gpc }\fi}} 
\newcommand{\hmpc}{{\ifmmode{h^{-1}{\rm Mpc}}\else{$h^{-1}$Mpc }\fi}} 
\newcommand{\hkpc}{{\ifmmode{h^{-1}{\rm kpc}}\else{$h^{-1}$kpc }\fi}}
\newcommand{\hMsun}{{\ifmmode{h^{-1}{\rm
        {M_{\odot}}}}\else{$h^{-1}{\rm{M_{\odot}}}$~}\fi}}  
\newcommand{\hmsun}{{\ifmmode{h^{-1}{\rm
        {M_{\odot}}}}\else{$h^{-1}{\rm{M_{\odot}}}$}\fi}}  
\newcommand{\Msun}{{\ifmmode{{\rm {M_{\odot}}}}\else{${\rm{M_{\odot}}}$}\fi}} 
\newcommand{\msun}{{\ifmmode{{\rm {M_{\odot}}}}\else{${\rm{M_{\odot}}}$}\fi}} 
\newcommand{\lya}{{Lyman$\alpha$~}}
\newcommand{\clara}{{\texttt{CLARA}}~}
\newcommand{\rand}{{\ifmmode{{\mathcal{R}}}\else{${\mathcal{R}}$ }\fi}} 
\newcommand{\hs}{{\hspace{1mm}}}
\newcommand{\muavg}{\vert\langle\cos\theta\rangle\vert}
% definition to produce a "less than or similar to" symbol
\def\lsim{~\rlap{$<$}{\lower 1.0ex\hbox{$\sim$}}}

% definition to produce a "greater than or similar to" symbol

% definition to produce a "greater than or similar to" symbol
\def\gsim{~\rlap{$>$}{\lower 1.0ex\hbox{$\sim$}}}

\begin{document}

\title[Bayesian halo concentration]{Measurement of dark matter halo concentrations using a bayesian approach}
\author[Poveda \& Forero-Romero]{
\parbox[t]{\textwidth}{\raggedright
  Christian Poveda$^{1}$ \&
  Jaime E. Forero-Romero$^{1}$
}
\vspace*{6pt}\\
$^{1}$Departamento de F\'{i}sica, Universidad de los Andes, Cra. 1
No. 18A-10, Edificio Ip, Bogot\'a, Colombia\\
}
\maketitle

\begin{abstract}
asd
\end{abstract}
\begin{keywords}
methods: numerical, galaxies: haloes, cosmology: theory, dark
matter
\end{keywords}


\section{Introduction}
\label{sec:introduction}


\citep{NFW}


\section{Basics of dark matter halos and the NFW profile}
\label{sec:basics}

The Navarro-Frenk-White density profile can be written as

\begin{equation}
\rho(r) = \frac{\rho_c\delta_c}{r/R_s(1+r/R_s)^2}, 
\end{equation}
%
where $\rho_c\equiv 3H^2/8\pi G$ is the Universe critical density,
$\delta_c$ is the halo dimensionless characteristic density. is a
normalization constant and $R_s$ is known as the scale radius, the
radius that marks the transition between the two power law behaviour
in the $\rho\propto r^{-1}$ for $r<R_s$ and $\rho\propto r^{-2}$ for
$r>R_s$. 

One can define the scale radius as a fraction of the virial radius,
$R_{vir}$,

\begin{equation}
R_s = R_{vir}/c,
\end{equation}

where $c>1$ is known as the concentration of the halo.

Taking the virial radius as the radius that encloses 200 times the
critical density, the concentration and the characteristic density are
related by 

\begin{equation}
\delta_c =\frac{200}{3}\frac{c^3}{[\ln(1+c)-c/(1+c)]}.
\end{equation}

Therefore, for a given value of the overdensity that defines a halo,
the density profile has only one free parameter: the concentration.


The total mass enclosed within a radius $r$ can be computed to be:

\begin{equation}
M(<r) = 4\pi\rho_c\delta_c  R_s^3\left[\ln \left
  (\frac{R_s+r}{r})\right) - \frac{r}{R_s+r}\right].
\end{equation}

If we now define dimensionless variables $x\equiv r/R_{\rm vir}$ and
$m\equiv M(<r)/M_{\rm vir}$...

profile can be normalized respect to the
virial mass and written in function of the normalized radius respect
to the virial radius $r=R/R_{vir}$ and the concentration $c$  



\begin{equation}
m_{\rm NFW}\left(r;c\right)=A\left[\log\left(1+rc\right)-\left(\frac{rc}{rc+1}\right)\right) 
\end{equation}
Where
\begin{equation}
A=\left[\log\left(1+c\right)-\left(\frac{c}{c+1}\right)\right]^{-1} 
\end{equation}

\section{Bayesian Method}
\label{sec:method}

In order to obtain a mass profile for each halo in funcion of the radius, the center of each halo must be calculated. However the center of mass will not give an accurate position for the center because some haloes are highly irregular, is more convenient to take the gravitational potential for each particle
\[
\phi\left(\boldsymbol{R}_{n}\right)=-\sum_{j\neq n}\frac{1}{\left\Vert \boldsymbol{R}_{n}-\boldsymbol{R}_{j}\right\Vert }
\]
Take $\boldsymbol{R}_{0}=\min\left(\phi\left(\boldsymbol{R}_{n}\right)\right)$ as the center then define new coordinates $\boldsymbol{R}_{n}^{\prime}=\boldsymbol{R}_{0}-\boldsymbol{R}_{n}$, this $\boldsymbol{R}_{0}$ will be in the most relevant region for the dynamics of each halo. Organizing the $\left\{ \boldsymbol{R}_{n}^{\prime}\right\} $ in crescent order, the accumulated mass for each $\boldsymbol{R}_{n}^{\prime}$ will be $M_{n}=nM$.

\begin{figure}[H]
\caption{FIXME: Mass profile of a halo}
\end{figure}

Then the virial radius is obtained and the particles that are beyond the virial radius are removed. After that both the mass and the radius get normalized respect to the mass and virial radius, we will call these new variables $m_{n}$ and $r_{n}$. The last part of the process consists in fitting the normalized NFW mass profile to the $m_{n}$ data using the Metropolis-Hastings algorithm to sample the Likelihood ${\cal L}\left(c\right)=\exp\left(-\chi^{2}\left(c\right)\right)$ where 
\[
\chi^{2}\left(c\right)=\sum\limits _{n}\left|m_{n}-m_{nfw}\left(r_{n},c\right)\right|^{2}
\]
And taking the values of $c$ where the Likelihood is maximum. Moreover, we can obtain a confidence interval for $c$ doing an histogram of the random walk done by the Metropolis-Hastings Algorithm. Finally we compare this results with the values obtaned by BDM, since the number of halos obtained by FOF and BDM is different, each BDM halo gets related with the nearest halo in FOF. We compute two concentrations, one where the overdensity limit is $360\rho_{back}$ (like in BDMV) an other where the overdensity limit is $740\rho_{back}$ (like in BDMW).

\section{Mock Data and Simulations}
\label{sec:data}
\section{Results}
\label{sec:results}

\subsection{Trial Problem with several Generated Halos}


Several halos with known values of $c$ were generated. These halos
were then processed to determine the acurracy of this method. The
number of particles and value of $c$ for each halo were generated
randomly.  
\begin{figure}[H]
\caption{FIXME: Test results}
\end{figure}



\subsection{Results from the simulations}


\subsection{What kind of implications (Low resolution simulations, other fitting algorithms)} 


\section{Discussion}
\label{sec:discussion}


\section{Conclusions}
\label{sec:conclusions}


\bibliographystyle{mn2e}
\bibliography{references}

\end{document}
